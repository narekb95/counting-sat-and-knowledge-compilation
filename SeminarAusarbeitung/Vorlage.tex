% Vorlage f�r Seminar-Ausarbeitungen
%
% Dateiname: Vorlage.tex
% zuletzt ge�ndert: 11. Februar 2016
% Autorin: Nicole Schweikardt

% Definition der Dokument-Klasse, in diesem Falle die Klasse SeminarAusarbeitung,
% die in der Datei SeminarAusarbeitung.cls bereitgestellt wird
\documentclass{SeminarAusarbeitung}

\usepackage{xcolor}
\title{Knowledge compilation and \#SAT}
\author{Narek Bojikian}
\seminar{Aktuelle Themen der theoretischen Informatik}
\semester{Wintersemester 2019/~2020}
\leitung{Prof.\ Dr.\ Christoph Berkholz} 
\institut{Institut f�r Informatik}
\universitaet{Humboldt-Universit�t zu Berlin}

\begin{document}

\maketitle

{{\color{red}\Huge todo} \color{blue}\Large 
\begin{itemize}
\item Add abstract.
\item Add stucture.
\item A paper written by? authored from? from? 
\item Check the date.
\item fill [todo] and check \\cite tags
\item add reference to the paper in abstract
\end{itemize}
}
\begin{abstract}
	In this report, we summarize the paper "Understanding the complexity of
	\#SAT using knowledge compilation" by Florent Capeli, 2018. We focus on
	the points presented and discussed during the seminar on 08.01.2020.
	More specifically, the paper is divided in to two parts. In the first of
	which, the author designs an efficient algorithm for the given problem
	using DPLL method, used to implement a practical solver for the \#SAT
	problem. In the second part, the author generalizes another schema, that
	had been used before to prove efficient algorithms for this problem on
	specific classes of formulas. He proves that this schema has a
	super-polynomial running time on this specific class of formulas.  We
	concentrate on the efficient algorithm presented in the paper and
	highlight its steps, and we suffice to list the results that yield a
	lower-bound on the running time of the other method.
\end{abstract}

\section{Background}
The \#SAT problem is the problem of finding the number of satisfying assignments
of a given formula.  The very related SAT problem, which targets the question
whether a given formula admits a satisfying assignment, has been one of the most
studied problems in theoretical computer science. The interest in this problem
increased drastically since the appearance of Cook-Levin theorem \cite{} [todo],
which states that SAT is an NP-hard problem.  This theorem has been prominent in
theoretical computer science ever since. One of the results of the theorem is
that a problem is NP-hard if it admits a polynomial reduction to SAT. This
inspired a follow up by Karp \cite{} [todo] and many other results that
initiated a new era of theoretical computer science. The interest in SAT and
NP-completeness however, influenced an increasing interest in a very related
class, namely \#P, which is the class of counting the number of certificates for
a balanced P problem. Clearly problems that are complete for \#P are at least as
hard as NP-complete problems, since NP is equivalent to the existential class of
P, namely $\exists$P. Interestingly, \#SAT problem is complete for \#SAT.  Again
this resulted in an increasing interest in the \#SAT problem. Among others,
counting the perfect matchings in a graph and computing the permanent of a
matrix are also complete for the \#P class and hence, any breakthrough in
solving \#SAT will have a great impact on the solvablility of these problems.

On the other hand, different approaches have been developed to tackle this
problem. On one side, there has been an interest in practical solvers, ones that
might not yield efficient bound in theory, but have proven themselves efficient
in practice. An example of which, is the DPLL solver. The idea of the solver is
to simplify the formula and to branch over variables to result in simpler
formulas. Careful cashing and choice of variables is essential for an efficient
solver. However, for many classes of formulas, this method can be quite
inefficient for carefully designed formulas. That is why this method had merely
been used in practice. On the other hand, Some approaches has been designed to
approach problems from the theoretical perspective. By restricting the classes
of formulas, some methods have proven efficient for solving \#SAT on these
formulas. For example in .. \cite{} [todo] .... However, the main goal of the
targeted paper is to show that the practical method can be more efficient even
in theory for some specific classes of formulas and thereby we should not
underestimate the method when approaching new classes of formulas. More
specifically, the author shows in this paper, that using the techniques from
DPLL, \#SAT on $\beta$-acyclic formulas can be done in Polynomial time,
meanwhile the standard theoretical technique has a super-polynomial lower-bound
on its running time.

A bit deeper into details, the theoretical techniques starts by building a
simpler formula equivalent to the input with useful properties and then solves
the problem on the resulting formula using dynamic programming. The paper shows
a subset of $\beta$-acyclic formulas such that any equivalent formula with that
specific set of properties has super-polynomial size and hence the method can
not be run in polynomial time. In contrast, the efficient method will also make
use of dynamic programming but it starts with finding an efficient order of the
vertices.

We start the report with a set of definitions that introduces all the concepts,
the structures and the ideas that will be used later in the report. Similar to
the presentation, we follow that with a brief introduction to $\beta$-acyclic
graph and list some useful properties of these graphs, that will show useful
later in the report. The third section contains the main lemmas that yield the
main result of the paper (own opinion) with brief sketch of the proofs. The
forth section is a list of results that yield the lower bounds on the
theoretical technique with according references. We conclude with a brief
summary of the results and what we learn from them.

\section{Preliminaries}
\subsection{Boolean formulas and assignments}
\subsection{Structured Formulas}
\subsection{Graphs and Hypergraphs}

\section{DPLL and solving \#SAT efficiently}

\section{Structured d-DNNF - Lower-bounds for the general approach}

\section{Conclusion}

\end{document}

