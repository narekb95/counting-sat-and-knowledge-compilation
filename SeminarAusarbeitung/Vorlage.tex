% Vorlage f�r Seminar-Ausarbeitungen
%
% Dateiname: Vorlage.tex
% zuletzt ge�ndert: 11. Februar 2016
% Autorin: Nicole Schweikardt

% Definition der Dokument-Klasse, in diesem Falle die Klasse SeminarAusarbeitung,
% die in der Datei SeminarAusarbeitung.cls bereitgestellt wird
\documentclass{SeminarAusarbeitung}

\usepackage{xcolor}
% Festlegen der Informationen zu Titel, Autor, Seminar, Semester, Seminarleitung, Institut, Universit�t
\title{Knowledge compilation and \#SAT}
\author{Narek Bojikian}
\seminar{Aktuelle Themen der theoretischen Informatik}
\semester{Wintersemester 2019/~2020}
\leitung{Prof.\ Dr.\ Christoph Berkholz} 
\institut{Institut f�r Informatik}
\universitaet{Humboldt-Universit�t zu Berlin}

% Hier beginnt das eigenliche Dokument
\begin{document}

% Erstellung des Dokument-Titels. Dabei werden die Informationen zu Titel, Autor etc. von oben verwendet.
% Das Aussehen des Titels ist in SeminarAusarbeitung.cls festgelegt
\maketitle

% Zusammenfassung/Abstract
{{\color{red}\Huge todo} \color{blue}\Large 
\begin{itemize}
\item Add abstract.
\item Add stucture.
\item A paper written by? authored from? from? 
\item Check the date.
\item add reference to the paper in abstract
\end{itemize}
}
\begin{abstract}
	In this report, we summarize the paper "Understanding the complexity of
	\#SAT using knowledge compilation" by Florent Capeli, 2018. We focus on
	the points presented and discussed during the seminar on 08.01.2020.
	More specifically, the paper is divided in to two parts. In the first of
	which, the author designs an efficient algorithm for the given problem
	using DPLL method, used to implement a practical solver for the \#SAT
	problem. In the second part, the author generalizes another schema, that
	had been used before to prove efficient algorithms for this problem on
	specific classes of formulas. In this paper, the author proves that this
	schema has a super-polynomial running time on this specific class of
	formulas.  We concentrate on the efficient algorithm presented in the
	paper and highlight its step, and we suffice to list the results that
	yield a lower-bound on the running time of the other
	method.
\end{abstract}

\section{Background}
\section{Preliminaries}
\section{An efficient algorithm - DPLL}
\section{Equivalent structured d-DNNF - Lower-bounds}
\section{Conclusion}

\end{document}

