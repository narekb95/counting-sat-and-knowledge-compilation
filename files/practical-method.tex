
\begin{frame}[t]{Lemmas on $\beta$-acyclic graphs}
	\begin{itemize}[<+->]
	\item Let $\mathcal{H}$ be a $\beta$-acyclic graph and $v_1, \dots v_n$ a $\beta$-elimination.
		\item For two edge $e, f \in \mathcal{H}$, $e < f$, if and only if $\max\{e \Delta f\}\in f$
		\item $\mathcal{H}^x_e$ denotes the subgraph of $\mathcal{H}$, that contain the edges $f$, such that there is a walk from $f$ to $e$ that goes only through edges smaller than $e$ and vertices smaller than (or equal to) $x$.
	\end{itemize}
	\begin{figure}[htpb]
		\centering
		\resizebox{.6\columnwidth}{!}{
		\begin{tikzpicture}
	\node (u) at (0, 0) {};
	\node (v) at (3, 3) {};

	\node (v1) at (1, 1) {};
	\node (v2) at (3, 2) {};
	\node (v3) at (5, 2) {};
	\node (v4) at (6, 3) {};
	\node (v5) at (8, 3) {};
	\begin{scope}[]
	\draw [draw=black, thick, fill=red!50] plot [smooth cycle, tension=.6] coordinates {
		($(v3) + (-1, -1)$)  ($(v3) + (1, 0)$)
		($(v4) + (1, 1)$) ($(v3) + (0, 1)$)
		};

	\draw [opacity=.8, draw=black, thick, fill=blue!70] plot [smooth cycle] coordinates {
			($(v1) + (0,-.5)$) ($(v1) + (-1,0)$) ($(v2) + (-.2,.5)$)
			($(v3) + (.5, .2)$) ($(v3) + (0, -.5)$) ($(v2) + (0, -1)$)};

%	\draw [fill=red!80] plot [smooth cycle] coordinates {
%			($(v3) + (.2,-.5)$) ($(v3) + (-.3,-.1)$)
%			($(v4) + (-.5, .7)$) ($(v4) + (.4, -.2)$)
%		};
	\draw [opacity=.5, draw=black, thick, fill=yellow!70] plot [smooth cycle] coordinates {
			($(v2) + (0,.5)$) ($(v3) + (-.3,.6)$) ($(v4) + (0,.5)$)
			($(v5) + (1,.5)$) ($(v5) + (.5,-.5)$) ($(v4) + (1,-.5)$) 
			($(v3) + (.5,-.5)$) ($(v2) + (-.2,-.5)$) 
		};
	\end{scope}

	\fill (v1) circle (0.1) node[above right] {$v_1$};
	\fill (v2) circle (0.1) node[right] {$v_2$};
	\fill (v3) circle (0.1) node[right] {$v_3$};
	\fill (v4) circle (0.1) node[below left] {$v_4$};
	\fill (v5) circle (0.1) node[right] {$v_5$};

	\node at (2,1.8) {$e_1$};
	\node at (6.5,4.1) {$e_2$};
	\node at (7.5,3.3) {$e_3$};
\end{tikzpicture}

		}
		\caption{Note that $e_2 \notin H^{v_2}_{e_3}$ meanwhile $e_2 \in H^{v_3}_{e_3}$}%
		\label{fig:name}
	\end{figure}
\end{frame}
\begin{frame}[t]{Lemmas on $\beta$-acyclic graphs}
	\begin{block}{Lemma (lemma 2)}
		For $x,y \in V(\mathcal{H}), x \leq y$ and for $e, f \in \mathcal{H}, e \leq f$,
			$$\text{if } V(\mathcal{H}^x_e)\cap V(\mathcal{H}^y_f)\cap V_{\leq x} \neq \emptyset,
			\text{ then } \mathcal{H}^x_e \subseteq \mathcal{H}^y_f.$$
			In particular, for all $y \in V(\mathcal{H})$,
		$$\text{if } e \in \mathcal{H}^y_f, \text{ then } \mathcal{H}^y_e \subseteq \mathcal{H}^y_f$$
	\end{block}
	\uncover<2->{
	Proof sketch. For $g \in \mathcal{H}^x_e$, there is a path from $g$ to $e$ using edges smaller than $e$ and vertices smaller than $x$.
	
	There is also a path from $e$ to $f$. Concatenate both paths to get a path from $g$ to $f$.
	}
\end{frame}

\begin{frame}[t]{Lemmas on $\beta$-acyclic graphs}
	\begin{block}{Lemma (lemma 4)}
		For $e, f \in \mathcal{H}, e\leq f$, If there exists a vertex $x \in V(\mathcal{H})$, such that $x \in e \cap f$, then $e \cap V_{\geq x} \subseteq f$.
	\end{block}
\end{frame}

\begin{frame}[t]{Lemmas on $\beta$-acyclic graphs}
	A path $(e_1, x_1, \dots e_{n+1})$ is called decreasing, if $e_i > e_{i+1}$ and $x_i > x_{i+1}$ for all $i$.
	\begin{block}{Lemma (lemma 5)}
		For $x \in V(\mathcal{H}), e \in \mathcal{H}$ and $f \in \mathcal{H}^x_e$, there exists a decreasing path from $e$ to $f$ going through vertices smaller than $x$.
	\end{block}
	Proof sketch. Any shortest path from $e$ to $f$ is decreasing. A path exists by definition.

\end{frame}

\begin{frame}[t]{Lemmas on $\beta$-acyclic graphs}
	\begin{block}{Theorem (theorem 3)}
		For every $x \in V(\mathcal{H})$ and $e \in \mathcal{H}, V(\mathcal{H}^x_e) \cap V_{\geq x} \subseteq e$
	\end{block}

	[figure]

	[why do we need this]
\end{frame}


\begin{frame}[t]{Solving \#SAT in $\beta$-acyclic graphs}
	\begin{itemize}
		\item For a clause $C$, we define the partial assignment $\tau_C$ over the variables of $C$ as the only assignment that does not satisfies $C$, i.e. for $x \in C$, $\tau_C(x) = 1$ if and only if $x$ appears as a negative literal in $C$.

		\item Let $F$ be a given $\beta$-acyclic CNF-formula and $v_1, \dots v_n$ be an elimination order of the variables in $F$. Let $\mathcal{H}$ be the hypergraph of $F$.
	\end{itemize}
	\begin{block}{Lemma (lemma 6)}
		Let $x \neq x_1 \in \mathrm{VAR}(F)$ and let $y$ be the predecessor of $x$ for $<$.  Let $e \in \mathcal{H}$ and $\tau : (e \cap V_{\geq x}) \rightarrow \{0, 1\}$. Then either $F^x_e[\tau] \equiv 1$ or there exists $U \subseteq \mathcal{H}^x_e$ such that 
		$$ F^x_e[\tau] \equiv \bigwedge\limits_{g \in U} F^y_g[\tau^y_{C_g}],$$
		where $C_g in F^x_e$ such that $\mathrm{C_g} = g$.
	\end{block}
\end{frame}

\begin{frame}[t]{Solving \#SAT in $\beta$-acyclic graphs}
		$$ F^x_e[\tau] \equiv \bigwedge\limits_{g \in U} F^y_g[\tau^y_{C_g}]$$
\end{frame}

\begin{frame}[t]{Solving \#SAT in $\beta$-acyclic graphs}
	\begin{block}{Corollary (corollary 7)}
		Let $x \neq x_1 \in \mathrm{VAR}(F)$ and let $y$ be the predecessor of $x$ for $<$. For every $C \in \mathcal{H}$, there exist $U_0, U_1 \subseteq \mathcal{H}^x_{\mathrm{VAR}(C)}$ such that
		$$F^x_{\mathrm{VAR}(C)}[\tau^x_C] \equiv 
		( x \land \bigwedge\limits_{g \in U_1} F^y_g[\tau^y_{C_g}]) \lor
		( \lnot x \land \bigwedge\limits_{g \in U_2} F^y_g[\tau^y_{C_g}])
		$$
	\end{block}
	Explanation.
\end{frame}

\begin{frame}[t]{Solving \#SAT in $\beta$-acyclic graphs}
	\begin{block}{Theorem (theorem 8)}
		Let $F$ be a $\beta$-acyclic CNF-formula. One can construct in polynomial time in $\mathrm{size}(F)$ a dec-DNNF $D$ of size $O(\mathrm(size(F))$ and fanin at most $|\mathcal{H}|$ computing F.
	\end{block}
\end{frame}

\begin{frame}[t]{concluding the practical method}
	\begin{itemize}[<+->]
		\item Exhaustive DPLL is a very-well used in practice method.
			\begin{itemize}
				\item Try to write $F$ as a decomposable conjunction. 
				\item[] \hspace{1cm}Solve independently on each and multiply the results.
				\item Choose a variable $x$.
				\item[] \hspace{1cm}Compute $\#F[x\mapsto 1] + \#F[x\mapsto 0]$.
			\end{itemize}
		\item The method makes use of cashing (choose what values to keep).
		\item Tries to find a good candidate for $x$.
		\item The previous dynamic programming is implicitly a run of DPLL. 
		\item The variables are chosen in e reversed $\beta$-elimination ordering. 
	\end{itemize}
	\uncover<10->{\begin{block}{conclusion}
		Exhaustive DPLL can yield efficient algorithms "theoretically", if we can find a good order to choose the variable (such an ordering must be computable in polynomial time) and a good method of cashing.
	\end{block}}
\end{frame}
