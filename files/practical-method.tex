\begin{frame}[t]{Lemmas on $\beta$-acyclic graphs}
	\begin{itemize}[<+->]
	\item Let $\mathcal{H}$ be a $\beta$-acyclic graph and $v_1, \dots v_n$ a $\beta$-elimination.
		\item For two edge $e, f \in \mathcal{H}$, $e < f$, if and only if $\max\{e \Delta f\}\in f$
		\item {\color{blue}$\mathcal{H}^x_e$} denotes the subgraph of $\mathcal{H}$, that contain the edges $f$, such that there is a walk from $f$ to $e$ that goes only through edges smaller than $e$ and vertices smaller than (or equal to) $x$.
	\end{itemize}
	\begin{figure}[htpb]
		\centering
		\resizebox{.6\columnwidth}{!}{
		\begin{tikzpicture}
	\node (u) at (0, 0) {};
	\node (v) at (3, 3) {};

	\node (v1) at (1, 1) {};
	\node (v2) at (3, 2) {};
	\node (v3) at (5, 2) {};
	\node (v4) at (6, 3) {};
	\node (v5) at (8, 3) {};
	\begin{scope}[opacity=.5]
	\draw [draw=black, fill=green!80] plot [smooth cycle] coordinates {
			($(v1) + (0,-.5)$) ($(v1) + (-1,0)$) ($(v2) + (-.2,.5)$)
			($(v3) + (.5, .2)$) ($(v3) + (0, -.5)$) ($(v2) + (0, -.7)$)};

	\draw [draw=black, fill=red!70] plot [smooth cycle] coordinates {
			($(v3) + (0,-.8)$) ($(v3) + (-.5,0)$)
			($(v4) + (-.3, .5)$) ($(v4) + (.4, -.2)$)
		};
	\draw [draw=black, fill=yellow!60] plot [smooth cycle] coordinates {
			($(v2) + (0,.5)$) ($(v3) + (0,.5)$) ($(v4) + (0,.5)$)
			($(v5) + (1,.5)$) ($(v5) + (.5,-.5)$) ($(v4) + (.5,-.5)$) 
			($(v3) + (.5,-.5)$) ($(v2) + (-.2,-.5)$) 
		};
	\end{scope}

	\fill (v1) circle (0.1) node[above right] {$v_1$};
	\fill (v2) circle (0.1) node[right] {$v_2$};
	\fill (v3) circle (0.1) node[right] {$v_3$};
	\fill (v4) circle (0.1) node[below left] {$v_4$};
	\fill (v5) circle (0.1) node[right] {$v_5$};

	\node at (2,1.8) {$e_1$};
	\node at (5,1) {$e_2$};
	\node at (7,3.4) {$e_3$};
\end{tikzpicture}

		}
		\uncover<4->{\caption{Note that $e_2 \notin \mathcal{H}^{v_2}_{e_3}$ meanwhile $e_2 \in \mathcal{H}^{v_3}_{e_3}$}}
		\label{fig:name}
	\end{figure}
\end{frame}
\begin{frame}[t]{Lemmas on $\beta$-acyclic graphs}
	\begin{block}{Lemma (lemma 2)}
		For $x,y \in V(\mathcal{H}), x \leq y$ and for $e, f \in \mathcal{H}, e \leq f$,
			$$\text{if } V(\mathcal{H}^x_e)\cap V(\mathcal{H}^y_f)\cap V_{\leq x} \neq \emptyset,
			\text{ then } \mathcal{H}^x_e \subseteq \mathcal{H}^y_f.$$
			In particular, for all $y \in V(\mathcal{H})$,
		$$\text{if } e \in \mathcal{H}^y_f, \text{ then } \mathcal{H}^y_e \subseteq \mathcal{H}^y_f$$
	\end{block}
	\uncover<2->{
	Proof sketch. For $g \in \mathcal{H}^x_e$, there is a path from $g$ to $e$ using edges smaller than $e$ and vertices smaller than $x$.
	
	There is also a path from $e$ to $f$. Concatenate both paths to get a path from $g$ to $f$.
	}
\end{frame}

\begin{frame}[t]{Lemmas on $\beta$-acyclic graphs}
	\begin{block}{Lemma (lemma 4)}
		For $e, f \in \mathcal{H}, e\leq f$, If there exists a vertex $x \in V(\mathcal{H})$, such that $x \in e \cap f$, then $e \cap V_{\geq x} \subseteq f$.
	\end{block}
	\uncover<2->{
	Proof sketch. If $y \in e \setminus f$ such that $y > x$, then $\mathcal{H}$ is not $\beta$-acyclic.
	}
\end{frame}

\begin{frame}[t]{Lemmas on $\beta$-acyclic graphs}
	A path $(e_1, x_1, \dots e_{n+1})$ is called decreasing, if $e_i > e_{i+1}$ and $x_i > x_{i+1}$ for all $i$.
	\begin{block}{Lemma (lemma 5)}
		For $x \in V(\mathcal{H}), e \in \mathcal{H}$ and $f \in \mathcal{H}^x_e$, there exists a decreasing path from $e$ to $f$ going through vertices smaller than $x$.
	\end{block}
	\uncover<2->{Proof sketch. Any shortest path from $e$ to $f$ is decreasing. A path exists by definition.}
\end{frame}

\begin{frame}[t]{Lemmas on $\beta$-acyclic graphs}
	\begin{block}{Theorem (theorem 3)}
		For every $x \in V(\mathcal{H})$ and $e \in \mathcal{H}, V(\mathcal{H}^x_e) \cap V_{\geq x} \subseteq e$
	\end{block}

	\begin{center}
		\resizebox{.5\columnwidth}{!}{
		\begin{tikzpicture}
	\node (u) at (0, 0) {};
	\node (v) at (3, 3) {};

	\node (v1) at (1, 1) {};
	\node (v2) at (3, 2) {};
	\node (v3) at (5, 2) {};
	\node (v4) at (6, 3) {};
	\node (v5) at (8, 3) {};
	\begin{scope}[opacity=.5]
	\draw [draw=black, fill=green!80] plot [smooth cycle] coordinates {
			($(v1) + (0,-.5)$) ($(v1) + (-1,0)$) ($(v2) + (-.2,.5)$)
			($(v3) + (.5, .2)$) ($(v3) + (0, -.5)$) ($(v2) + (0, -.7)$)};

	\draw [draw=black, fill=red!70] plot [smooth cycle] coordinates {
			($(v3) + (0,-.8)$) ($(v3) + (-.5,0)$)
			($(v4) + (-.3, .5)$) ($(v4) + (.4, -.2)$)
		};
	\draw [draw=black, fill=yellow!60] plot [smooth cycle] coordinates {
			($(v2) + (0,.5)$) ($(v3) + (0,.5)$) ($(v4) + (0,.5)$)
			($(v5) + (1,.5)$) ($(v5) + (.5,-.5)$) ($(v4) + (.5,-.5)$) 
			($(v3) + (.5,-.5)$) ($(v2) + (-.2,-.5)$) 
		};
	\end{scope}

	\fill (v1) circle (0.1) node[above right] {$v_1$};
	\fill (v2) circle (0.1) node[right] {$v_2$};
	\fill (v3) circle (0.1) node[right] {$v_3$};
	\fill (v4) circle (0.1) node[below left] {$v_4$};
	\fill (v5) circle (0.1) node[right] {$v_5$};

	\node at (2,1.8) {$e_1$};
	\node at (5,1) {$e_2$};
	\node at (7,3.4) {$e_3$};
\end{tikzpicture}

		}
	\end{center}
	\uncover<2->{Proof sketch. Prove that all edges of a decreasing path are subsets of the first edge by induction over the length of the path.}

	\uncover<3->{Intuitively, this allows us to use dynamic programming, since all variables in $\mathcal{H}^x_e$ not contained in $e$ are smaller than $x$.}

\end{frame}


\begin{frame}[t]{Solving \#SAT in $\beta$-acyclic graphs}
	\begin{itemize}[<+->]
	\item The hypergraph of a CNF-formula:
	\item[]\hspace{1cm}Variables of each clause correspond to an edge.
	\item[]\hspace{1cm}Two clauses might correspond to the same edge.
	\item A $\beta$-acyclic CNF-formula F is given.
	\item Let $\mathcal{H}$ be the hypergraph of $F$.
	\item Let $v_1, \dots v_n$ be an elimination order in $\mathcal{H}$.
	\item Let {\color{blue}$F^x_e$} be the subformula of $F$ that corresponds to $\mathcal{H}^x_e$,

	\hspace{1cm}i.e. $C \in F^x_e$, if $\mathrm{VAR}(C) \in \mathcal{H}^x_e$.
	\item For a clause $C$, the partial assignment {\color{blue}$\tau_C$} is defined over $\mathrm{VAR}(C)$ as the only assignment that does not satisfy $C$,
	\item[]\hspace{1cm}i.e. for $x \in C$, $\tau_C(x) = 1$ if and only if $x$ appears as a negative literal in $C$.
	\item We define ${\color{blue}\tau^x_C} := \tau_{C|\geq x}$,
	\item[]\hspace{1cm}{\color{blue} $F[\tau^x_C]$} results from $F$ by removing all variables greater than $x$ from each clause.
	\end{itemize}
\end{frame}
\begin{frame}[t]{Solving \#SAT in $\beta$-acyclic graphs}
	\begin{block}{Lemma (lemma 6)}
		Let $x \neq x_1 \in \mathrm{VAR}(F)$ and let $y$ be the predecessor of $x$ for $<$. Let $e \in \mathcal{H}$ and $\tau : (e \cap V_{\geq x}) \rightarrow \{0, 1\}$. Then either $F^x_e[\tau] \equiv 1$ or there exists $U \subseteq \mathcal{H}^x_e$ such that 
		$$ F^x_e[\tau] \equiv \bigwedge\limits_{g \in U} F^y_g[\tau^y_{C_g}],$$
		where $C_g$ is some clause in $F^x_e$ such that $\mathrm{VAR}(C_g) = g$.
		
		Moreover, all and-gates are decomposable and $U$ can be computed in polynomial time.
	\end{block}
\end{frame}
\begin{frame}[t]{Solving \#SAT in $\beta$-acyclic graphs}
	\vspace{-1cm}
	\begin{center}
		$$ F^x_e[\tau] \equiv \bigwedge\limits_{g \in U} F^y_g[\tau^y_{C_g}],$$
	\end{center}
	Proof sketch.
	\begin{itemize}[<+->]
		\item Let $A$ be the set edges $g$, where $\tau \not \models C_g$ for some {\color{gray} \textit{corresponding}} $C_g$.
		\item For each clause $C$ such that $\mathrm{VAR}(C) \notin A$ we have $\tau \models C$.
		\item Choose $U$ as the set of "maximal" edges $g \in A$,
		\item For each $f \in A$, there is $g \in U$ such that $f \in \mathcal{H}^y_g$.
		\item[]\hspace{1cm}i.e. $g \not \subseteq \mathcal{H}^y_f$ for all $f \in A$, $f \neq g$.
		\item $U$ can be computed in polynomial time.
		\item The edges in $U$ are pairwise disjoint. Hence, the and-gate is  decomposable. 
	\end{itemize}
\end{frame}

\begin{frame}[t]{Solving \#SAT in $\beta$-acyclic graphs}
	\begin{block}{Corollary (corollary 7)}
		Let $x \neq x_1 \in \mathrm{VAR}(F)$ and let $y$ be the predecessor of $x$ for $<$. For every $C \in \mathcal{H}$, there exist $U_0, U_1 \subseteq \mathcal{H}^x_{\mathrm{VAR}(C)}$ such that
		$$F^x_{\mathrm{VAR}(C)}[\tau^x_C] \equiv 
		( x \land \bigwedge\limits_{g \in U_1} F^y_g[\tau^y_{C_g}]) \lor
		( \lnot x \land \bigwedge\limits_{g \in U_2} F^y_g[\tau^y_{C_g}]).
		$$
		Moreover, all conjunctions are decomposable and $U_0, U_1$ can be computed in polynomial time.
	\end{block}

	\pause
	Proof sketch. Let $\tau_1 := \tau^x_C \cup \{x \mapsto 1\}$ and $\tau_0 := \tau^x_C \cup \{x \mapsto 0\}$.
	$$F^x_{\mathrm{VAR}(C)}[\tau^x_C] = (x \land F^x_{\mathrm{VAR}(C)}[\tau_1]) \lor (\lnot x \land F^x_{\mathrm{VAR}(C)}[\tau_0])$$	
		Apply lemma 6 on each of the terms.
\end{frame}

\begin{frame}[t]{Solving \#SAT in $\beta$-acyclic graphs}
	\begin{block}{Theorem (theorem 8)}
		Let $F$ be a $\beta$-acyclic CNF-formula. One can construct in polynomial time in $\mathrm{size}(F)$ a dec-DNNF $D$ of size $O(\mathrm(size(F))$ and fanin at most $|\mathcal{H}|$ computing F.
	\end{block}
	\pause
	Proof sketch.
	Let $D_i$ be a dec-DNNF of fanin $|\mathcal{H}|$ at most such that for each $e \in \mathcal{H}, C \in F$ such that $\mathrm{VAR}(C) = e$ and $j \leq i$, there exists a gate in $D_i$ computing $F^{x_j}_e[\tau^{x_j}_c]$.
	\pause
	\begin{itemize}[<+->]
		\item Construct $D_i$ inductively over $i$.
		\item For an edge $e$, distinguish the cases whether $v_i \in e$ or not.
		\item For $i=1$,
		\item[]\hspace{1cm}$v_1 \notin e$: $F^{v_1}_e[\tau_C] = 0$.
		\item[]\hspace{1cm}$v_1 \in e$: $F^{v_1}_e \in \{1, v_1, \lnot v_1\}$.
	\end{itemize}
\end{frame}
\begin{frame}[t]{Proof sketch of theorem 8}
	\begin{itemize}[<+->]
		\item If $v_{i+1} \notin e$, then $F^{x_{i+1}}_e = F^{x_i}_e$.

		\item If $v_{i+1} \in e$, then by corollary 7 we can compute $F^{x_{i+1}}_e$ by adding a decision gate and (two) fanin at most $\mathcal{H}$ decomposable and-gates.
		
		\item[]\hspace{1cm}For each other terms from corollary 7 there is already a gate in $D_i$ that computes this term by induction hypothesis. 
		\item For $e = \max\mathcal{H}$ and a clause $C$ such that $\mathrm{VAR}(C) = e$, we have $\mathcal{H}^{v_n}_e = \mathcal{H}$ and $\tau^{x_n}_{\mathrm{VAR}(C)} = \emptyset$, hence there is a gate in $D_n$ computing $F^{x_n}_e[\tau_C] = F$.
		\item We add at most 7 gates per edges per vertex.
	\end{itemize}

\end{frame}

\begin{frame}[t]{Example}
	\begin{center}
		\resizebox{.45\columnwidth}{!}{
		\begin{tikzpicture}
	\node (u) at (0, 0) {};
	\node (v) at (3, 3) {};

	\node (v1) at (1, 1) {};
	\node (v2) at (3, 2) {};
	\node (v3) at (5, 2) {};
	\node (v4) at (6, 3) {};
	\node (v5) at (8, 3) {};
	\begin{scope}[opacity=.5]
	\draw [draw=black, fill=green!80] plot [smooth cycle] coordinates {
			($(v1) + (0,-.5)$) ($(v1) + (-1,0)$) ($(v2) + (-.2,.5)$)
			($(v3) + (.5, .2)$) ($(v3) + (0, -.5)$) ($(v2) + (0, -.7)$)};

	\draw [draw=black, fill=red!70] plot [smooth cycle] coordinates {
			($(v3) + (0,-.8)$) ($(v3) + (-.5,0)$)
			($(v4) + (-.3, .5)$) ($(v4) + (.4, -.2)$)
		};
	\draw [draw=black, fill=yellow!60] plot [smooth cycle] coordinates {
			($(v2) + (0,.5)$) ($(v3) + (0,.5)$) ($(v4) + (0,.5)$)
			($(v5) + (1,.5)$) ($(v5) + (.5,-.5)$) ($(v4) + (.5,-.5)$) 
			($(v3) + (.5,-.5)$) ($(v2) + (-.2,-.5)$) 
		};
	\end{scope}

	\fill (v1) circle (0.1) node[above right] {$v_1$};
	\fill (v2) circle (0.1) node[right] {$v_2$};
	\fill (v3) circle (0.1) node[right] {$v_3$};
	\fill (v4) circle (0.1) node[below left] {$v_4$};
	\fill (v5) circle (0.1) node[right] {$v_5$};

	\node at (2,1.8) {$e_1$};
	\node at (5,1) {$e_2$};
	\node at (7,3.4) {$e_3$};
\end{tikzpicture}

		}
		$$F = \{\{\overline{v_1}, v_2, v_3\}, \{\overline{v_3}, v_4\}, \{v_2, v_3, \overline{v_4}, \overline{v_5}\}\}$$
	\end{center}
	\vspace{1cm}\hspace{7cm}The rest on the blackboard..
\end{frame}
\begin{frame}[t]{concluding the practical method}
	\begin{itemize}[<+->]
		\item Exhaustive DPLL is a very-well used in practice method.
			\begin{itemize}
				\item Try to write $F$ as a decomposable conjunction. 
				\item[] \hspace{1cm}Solve independently on each and multiply the results.
				\item Choose a variable $x$.
				\item[] \hspace{1cm}Compute $\#F[x\mapsto 1] + \#F[x\mapsto 0]$.
			\end{itemize}
		\item The method makes use of cashing (choose what values to keep).
		\item Tries to find a good candidate for $x$.
		\item The previous dynamic programming is implicitly a run of DPLL. 
		\item The variables are chosen in e reversed $\beta$-elimination ordering. 
	\end{itemize}
	\uncover<10->{\begin{block}{conclusion}
		Exhaustive DPLL can yield efficient algorithms "theoretically", if we can find a good order to choose the variable (such an ordering must be computable in polynomial time) and a good method of cashing.
	\end{block}}
\end{frame}
